\documentclass[12pt]{article}
\usepackage{array}
\usepackage{bbm}
\usepackage{enumitem}
\usepackage[margin=0.5in]{geometry}
\usepackage{amsmath,amsthm,amssymb}
\usepackage{caption}
\usepackage{scrextend}
\usepackage{tikz}
\usepackage{tikz-cd}
\usepackage{titlesec}
\usepackage{bm}
\usetikzlibrary{positioning}
\usepackage[obeyspaces]{url}
\usepackage{caption}
\captionsetup{font = footnotesize}
\usepackage[
backend=biber,
style=numeric,
sorting=ynt
]{biblatex}
\addbibresource{biblio.bib}
\captionsetup{font = footnotesize}
\makeatother
\makeatletter
\pagestyle{empty}

\begin{document}
\textbf{Schuyler Smith \hfill BCB 568 Spring 2018\hfill Homework 6}\\

\section{T92 model of DNA evolution}
\begin{enumerate}
  \item If we set $\kappa=1$ and $g=0.5$, indicative of universal transition-transversion rates and homogeneous nucleotide frequencies, then the $Q$-matrix resolves to:
    \[
     Q= \begin{pmatrix}
     -  & 1/4 & 1/4 & 1/4\\ 
     1/4 & -  & 1/4 & 1/4 \\ 1/4 & 1/4 & -  & 1/4\\ 1/4 & 1/4 & 1/4 & - 
     \end{pmatrix}
    \]
  With the product of $Q$ and $\mu$, the overall rate of substitution, we can obtain the Jukes-Cantor model.\\
  
\item If $q_{ij} > 0$ for all $i \neq j$, then it is defined to have an existing stationary distribution ($\pi$). For the T92 model $Q$,  $\kappa > 0 < g<1$, which indicates that it does in-fact have a stationary distribution.
  
\item Given the information we have, it seems most intuitive to use the GC-content to model the distribution. To be stationary it must satisfy $\pi^T Q =0^T$, so the distribution of $\pi_{(A,G,C,T)}=((1-g)/2,\ g/2,\ g/2,\ (1-g)/2)$\\
  \[
 (0,0,0,0)= \begin{pmatrix}
 \frac{1-g}{2},  & \frac{g}{2}, & \frac{g}{2}, & \frac{1-g}{2}\\ 
 \end{pmatrix}
 \begin{pmatrix}
 -(\kappa g+1)/2  & \kappa g/2 & g/2 & (1 - g)/2\\ 
 \kappa(1 - g)/2 & -(\kappa-\kappa g+1)/2 & g/2 & (1 - g)/2 \\ 
 (1 - g)/2 & g/2 & -(\kappa- \kappa g+1)/2  & \kappa(1 - g)/2\\ 
 (1 - g)/2 & g/2 & \kappa g/2 & -(\kappa g+1)/2 \\
 \end{pmatrix}
 \]\\
   Since it is symmetrical, the equations match for A-T and G-C.
  \[
  \pi_{A,T}=\frac{-(\kappa g+1)(1-g)}{4} + \frac{g\kappa(1 - g)}{4}+ \frac{g(1 - g)}{4} + \frac{(1-g)^2}{4}=0
  \]
  \[
  \pi_{G,C}=\frac{\kappa g(1-g)}{4} - \frac{g(\kappa- \kappa g+1)}{4}+ \frac{g^2}{4} + \frac{g(1 - g)}{4}=0
  \]\\
  Thereby $\pi=((1-g)/2,g/2,g/2,(1-g)/2)$ is the stationary distribution of this T92 model.
  
\item If $\pi_i q_{ij}=\pi_j q_{ji}$ for all $i \neq j$ then it is shown to be time-reversible. This can be shown by equating all transition-transversion pairs:
  \[
  \text{AC - CA} = \frac{(1-g)}{2} \frac{g}{2}=  \frac{g}{2} \frac{(1-g)}{2}
  \]
  \[
  \text{AG - GA} = \frac{(1-g)}{2} \frac{\kappa g}{2}=  \frac{g}{2} \frac{\kappa (1-g)}{2}
  \]
  \[
  \text{AT - TA} = \frac{(1-g)}{2} \frac{(1-g)}{2}=  \frac{(1-g)}{2} \frac{(1-g)}{2}
  \]
  \[
 \text{CG - GC} = \frac{g}{2} \frac{(-\kappa + \kappa g -1)}{2}=  \frac{(-\kappa + \kappa g -1)}{2} \frac{g}{2}
  \]
  \[
 \text{CT - CT} = \frac{g}{2} \frac{(\kappa (1-g))}{2}=  \frac{(\kappa g)}{2} \frac{(1-g)}{2}
  \]
  \[
  \text{GT - GT} =\frac{g}{2} \frac{(1-g)}{2}=  \frac{(1- g)}{2} \frac{g}{2}
  \]

  \item The overall rate of substitution, $\mu$, is:
  
    \mu= \sum_{i \in (A,G,C,T)}\sum_{j \neq i} \pi_i q_{ij}= \sum_{i \in (A,G,C,T)} \pi_i (-q_{ii})
    = \frac{1-g}{2}  \frac{(\kappa g+1)}{2}+ \frac{g}{2} \frac{(\kappa-\kappa g+1)}{2} + \frac{g}{2} \frac{(\kappa-\kappa g+1)}{2} + \frac{1-g}{2}  \frac{(\kappa g+1)}{2}$
  
  \[
    \mu= -\kappa g^2+\kappa g + \frac{1}{2}
  \]

  
  \item The $Q$-matrix was made using $q_{ij}=\nu_i p_{ij}, i\neq j$ with $\nu_i = -q_{ii}$. If we use the $Q$-matrix without respect to time, we obtain discrete transitions, and the $P$-matrix can be formulated as $p_{ij}=q_{ij}/-q_{ii}$, exploiting the properties of the diagonal in the $Q$. Thus, we get 
  
   \[
 P= 
 \begin{pmatrix}
 0  & \frac{\kappa g}{(\kappa g+1)} & \frac{g}{(\kappa g+1)}& \frac{(1-g)}{(\kappa g+1)}\\ 
 \frac{\kappa(1 - g)}{\kappa-\kappa g+1} & 0 & \frac{g}{\kappa-\kappa g+1} & \frac{(1 - g)}{\kappa-\kappa g+1} \\ 
 \frac{(1 - g)}{\kappa-\kappa g+1} & \frac{g}{\kappa-\kappa g+1} & 0  & \frac{\kappa(1 - g)}{\kappa-\kappa g+1}\\ 
 \frac{(1-g)}{(\kappa g+1)} & \frac{g}{(\kappa g+1)} & \frac{\kappa g}{(\kappa g+1)} & 0 \\
 \end{pmatrix}
 \]
\end{enumerate}
\end{document}
