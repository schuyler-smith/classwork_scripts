\documentclass[12pt]{article}

 
\usepackage{array} 
\usepackage{enumitem}
\usepackage[margin=0.75in]{geometry} 
\usepackage{amsmath,amsthm,amssymb}
\usepackage{caption}
\usepackage{scrextend}
\usepackage{tikz}
\usepackage{tikz-cd}
\usepackage{permute}
\usepackage{titlesec}
\usetikzlibrary{positioning}
\newtheorem{theorem}{Theorem}
\numberwithin{theorem}{subsection}
\newtheorem*{theorem*}{Theorem}
\theoremstyle{definition}
\newtheorem{definition}{Definition}[section]
\numberwithin{definition}{subsection}
\newtheorem{example}{Example}[section]
\newtheorem{corollary}{Corollary}[section]
\numberwithin{corollary}{subsection}
\theoremstyle{remark}
\newtheorem*{remark}{Remark}
\graphicspath{ {images/} }
\usepackage[
backend=biber,
style=numeric,
sorting=ynt
]{biblatex}
\addbibresource{biblio.bib}
\DeclareMathOperator{\lcm}{lcm}
\newcommand{\C}{\mathbb{C}}
\newcommand{\N}{\mathbb{N}}
\newcommand{\Q}{\mathbb{Q}}
\newcommand{\R}{\mathbb{R}}
\newcommand{\Z}{\mathbb{Z}}
\newcommand\numberthis{\addtocounter{equation}{1}\tag{\theequation}}
\newcommand{\LEMMA}[1]{\textbf{Lemma #1.\ }}
\newcommand{\divides}{\bigm|}
\newcommand{\ndivides}{%
  \mathrel{\mkern.5mu % small adjustment
    % superimpose \nmid to \big|
    \ooalign{\hidewidth$\big|$\hidewidth\cr$\nmid$\cr}%
  }%
}
\newcommand{\ubt}[2]{\underbrace{#1}_{\mbox{\small #2}}}
\newcommand{\ubm}[2]{\underbrace{#1}_{#2}}
\newcommand{\aut}[1]{\operatorname{Aut}(#1)}
\newcommand{\groupchar}[2]{#1\,\operatorname{char}\,#2}
\setcounter{secnumdepth}{4}
\titleformat{\paragraph}{\normalfont\normalsize\bfseries}{\theparagraph}{1em}{}
\titlespacing*{\paragraph}{0pt}{3.25ex plus 1ex minus .2ex}{1.5ex plus .2ex}


\begin{document}
\setlength{\parindent}{0pt}
\title{BCB 568 Homework Assignment 1}
\author{Schuyler Smith}
\date{January 25, 2018}
 
\maketitle

\begin{enumerate}[label=\textbf{\arabic*.}]

    \item $\bullet$ All nucleotides are independent.\\
    $\bullet$ All nucleotides are randomly distributed.\\
    $\bullet$ All nucleotides occur with equal probability.
    
    \item 
        \begin{enumerate}[label = \textbf{\alph*.}]
            \item If we can use $m:\Omega_{\text{codon}} \to \Omega_{aa}$ to represent the relation of a specific codon to its amino acid, then $\mathcal{C}_{aa}$ is the set of all codons that translate to amino acid $aa$ (i.e. $\mathcal{C}_F$ = \{TTT, TTC\}). For our model to work we need to make the same assumptions from problem-1, which allows us to assume that when an amino acid is observed then any of the codons comprising $\mathcal{C}_{aa}$ are equally likely to be what we are observing. That would mean the probability for any codon considering amino acid probabilities is

                \begin{align*}
                    P(C = c)    &= P(C = c \mid A = m(c))P(A = m(c)) \\
                                &= \frac{1}{|\mathcal{C}_{m(c)}|}\cdot P(A = m(c)) \\
                                &=\frac{P(A = m(c))}{|\mathcal{C}_{m(c)}|}
                \end{align*}
                
            If we sum the probabilities for all codons for any amino acid it takes us back to the original probability for amino acids, P(A = a):
                \begin{align*}
                    \sum_{c \in \mathcal{C}_{aa}} P(C = c) = \sum_{c \in \mathcal{C}_{aa}} \frac{P(A = m(c))}{|\mathcal{C}_{m(c)}|} = \sum_{c \in \mathcal{C}_{aa}} \frac{P(A = a)}{|\mathcal{C}_{aa}|} = |\mathcal{C}_{aa}|\cdot\frac{P(A = a)}{|\mathcal{C}_{aa}|} = P(A = a), 
                \end{align*}

            
            \item If we again use the same assumptions from problem-1, we know that $P(A = a) = |\mathcal{C}_{aa}| / 64$.  So for any $c \in \Omega_{\text{codon}}$, it can be modeled as
                \begin{align*}
                    P(C = c) = \frac{P(A =m(c))}{|\mathcal{C}_{m(c)}|} = \frac{|\mathcal{C}_{m(c)}|}{64}\cdot\frac{1}{|\mathcal{C}_{m(c)}|} = \frac{1}{64}
                \end{align*}

        \end{enumerate}
        
    \item 
        \begin{enumerate}[label = \textbf{\alph*.}]
            \item An ORF has been defined as a sequence of codons beginning with a START-codon and terminated by a STOP-codon. In this way, the shortest ORF, in theory, would be of length 1, for the methionine from the START-codon. Essentially we want to say that the length $L$ of a randomly generated ORF is going to be dependant on the probability $p$ of observing a STOP-codon where the only alternative is finding an amino acid codon. The $p$ we will define so that it will be equal to the probability of seeing the nucleotides in their sequence to compose one of the three STOP-codons

            	\begin{align*}
                    p = P(TAA) + P(TAG) + P(TGA)
                \end{align*}
            The probability of these codons is defined by the probability of the nucleotides $q_{nucleotide}$ such that

            	\begin{align*}
                    p 	&= q_{T}q_{A}q_{A} + q_{T}q_{A}q_{G} + q_{T}q_{G}q_{A} \\
                    	&= 2q_{T}q_{A}q_{G} + q_{T}{q_{A}}^2
                \end{align*}

            If we define finding a STOP-codon as a success and finding an amino acid codon as a failure we can model $p$ as a geomettric random variable thusly
       
                \begin{align*}
                    P(L = l) = p(1 - p)^{l - 1}
                \end{align*}

            \item $Loxodonta$ $africana$ (african elephants) have mean GC content of $q_{GC} = 0.477$. To sufficiently answer the question we need to make the assumption that $q_{C} = q_{G} = 0.477/2 = 0.2385$ and that $q_{A} = q_{T} = (1-0.477)/2 = 0.2615$. We can use these ratios for the equation we got for part (a) and have 

            	\begin{align*}
                    p 	&= 2(0.2385)(0.2615)^2 + 0.2615^3 \\
                    	&= 0.0505
                \end{align*}

            So if we model $L$ as we did before and looking for a cutoff $l$ to where we want to call the cutoff for what is most likely a true gene we can write the cumulative density function as 

                \begin{align*}
                    P(L \leq l) = 1 - (1 - p)^l,
                \end{align*}
                
            If we set our false possitive level $\alpha$ and say that an ORF is not a true gene if $P(L \geq l) \leq \alpha$. At $\alpha = 0.05$, we see 

                \begin{align*}
                    P(L \geq l) &= (1 - p)^1 = 0.05.
                    l = \frac{\ln{0.05}}{\ln{(1 - p)}} = \frac{\ln{0.05}}{\ln{0.9495}} = 57.81 \approx 58.
                \end{align*}

            So using these estimates we would say that an ORF with $l \leq 58$ is not likely to be a true gene. The reasoning being that if it exists at a length longer than what is statistically likely by random chance, the reason must be for a biological purpose.
            
            
            \item Though the assignment comments that GC content may range anywhere from 20-80\% I chose to model GC content at 0.42-0.56 with incrememnts of 0.02. I thought this would be interesting as it demonstrates the difference even a slight deviation from equilibrium can make. It turned out that it was not as drastic as I thought, but still has some interesting points. The graph is in another file, taken from an R graph output.
        \end{enumerate}
        
    \item 
        \begin{enumerate}[label = \textbf{\alph*.}]
            \item Isochores, $I$, have a GC content of $q_{GC}^I$ and $q_A^I, q_G^I, q_C^I$, $q_T^I$ asthe proportions each nucleoptide in $I$.  For an ORF $o$, $o \in I$ denotes that $o$ is contained in isochore $I$. To resolve our model we need to make additional assumptions\\\\
                    $\bullet$ Within each isochore $I$, nucleotides are independently randomly distributed \\
                    $\bullet$ Genes and isochores are adjacent, no introns or "junk" DNA \\
                    $\bullet$ Each ORF is found entirely in a single isochore \\
            A revised model for ORF length that accounts for isochores and using the geometric random variable model for STOP-codons we applied earlier here as $P(L = l | o \in I)$
                \begin{align*}
                    P(L = l) &= P(L = l | o \in L_1)P(o \in L_1) + P(L = l | o \in L_2)P(o \in L_2)\\
                             &+ P(L = l | o \in H_1)P(o \in H_1) + P(L = l | o \in H_2)P(o \in H_2)\\
                             &+ P(L = l | o \in H_3)P(o \in H_3),
                \end{align*}
            
            \item From the table of isochores we were given, on average, isochores have abnormally low GC-content. Lower GC-content precludes STOP-codons from being found. Knowing that, without accounting for isochores, our previous estimates for $L$ of ORFs would be too conservative and that they acceptable $l$ for concluding a true gene would be longer than estimated. 
        \end{enumerate}

    
  

\end{enumerate}

\printbibliography

\end{document}
\grid
