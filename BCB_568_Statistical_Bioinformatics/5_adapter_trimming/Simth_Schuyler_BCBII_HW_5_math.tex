\documentclass[10pt]{article}

\usepackage{array} 
\usepackage{bbm}
\usepackage{clrscode3e}
\usepackage{enumitem}
\usepackage[margin=0.75in]{geometry} 
\usepackage{amsmath,amsthm,amssymb}
\usepackage{caption}
\usepackage{scrextend}
\usepackage{tikz}
\usepackage{adjustbox}
\usepackage{blindtext}
\usepackage{mathtools}
\usepackage{tikz-cd}
\usepackage{titlesec}
\usepackage{bm}
\usetikzlibrary{positioning}
\usepackage[obeyspaces]{url}
\usepackage{rotating}

\usepackage{caption}
\captionsetup{font = footnotesize}

\setenumerate{listparindent=\parindent}


\graphicspath{ {images/} }

\begin{document}
Schuyler Smith \hfill BCB 568 Spring 2018\hfill Homework \#5

\hrulefill

\begin{enumerate}[label = \textbf{\arabic*.}]
    \item Let $X$ be the observed sequence.  Define $Y$ as the true sequence, including the random leading nucleotides, barcode, and sample sequence, and let $Y_j$ be the $j$th true nucleotide sequenced as observed nucleotide $X_j$.  We are given that $z_{i2} \in \{0, 1, 2, 3\}$, and that this corresponds to the realization of the nucleotides at positions 35 and 42.  We enumerate the possible combinations of the bases at positions 35 and 42 in the following way:
                \begin{figure}[h]
                         \begin{center}
                            \begin{tabular}{ |c|c|c| } 
                                \hline
                                $z_{i2}$ & Y & R\\
                                \hline
                                0 & C & A \\
                                \hline
                                1 & C & G\\
                                \hline
                                2 & T & A\\
                                \hline
                                3 & T & G\\
                                \hline
                            \end{tabular}
                        \end{center}
                        \label{fig:alpha_1}
                \end{figure}
                
        We can convert the value of $z_{i2}$ to the true nucleotides at positions 35 and 42 as such:
            \begin{gather*}
            Y_{35} = \left\{\begin{array}{ll}
                            C & \textnormal{if }  z_{i2} < 2, \\
                            T & \textnormal{otherwise}
                        \end{array}\right.
            \qquad
            Y_{42} = \left\{\begin{array}{ll}
                            A & \textnormal{if }  z_{i2}\, \%\, 2 = 0, \\
                            G & \textnormal{otherwise}
                        \end{array}\right.
            \end{gather*}
    
    Here, \% is the modulus operator (\textit{i.e.}, $2\, \%\, 2 = 2\!\!\mod{2} = 0)$.  Additionally, define
        \begin{align*}
            \mathcal{I}_P \coloneqq \{1, 2, \dots, 18, 28, \dots, 34, 36, \dots, 41, 43,\dots, 52 \}
        \end{align*} and $\mathcal{I}_B \coloneqq \{19, 20, \dots, 27\}$. Then, picking some sequence $X$ so that we can drop the index $i$, the probability of observing base $x$ at index $j$ is
        \begin{align}\label{eq:p1}
            \qquad p_{jz} &= P(X_j = x |\, \bm{Z} = \bm{z}) =\nonumber\\
                            &= (q_{Bx})^{\mathbbm{1}\{j \leq z_1\}}\times\bigg[\delta_j^{\mathbbm{1}\{x = Y_{j - z_1}\}}\big((1 - \delta_j)\gamma_{Y_{j - z_1}x}\big)^{\mathbbm{1}\{x \neq Y_{j - z_1}\}}\bigg]^{\mathbbm{1}\{j - z_1 \in \mathcal{I}_P\}}\nonumber\\
                            &\,\times (q_{Bx})^{\mathbbm{1}\{j - z_1\,\in \mathcal{I}_B\}}\times(q_{Sx})^{\mathbbm{1}\{53\, \leq\, j - z_1\}}\nonumber\\
                            &\,\times\bigg[\Big[ \delta_j^{\mathbbm{1}\{x = C\}}\big((1 - \delta_j)\gamma_{Cx}\big)^{\mathbbm{1}\{x \neq C\}}\Big]^{\mathbbm{1}\{z_2 < 2\}}\Big[ \delta_j^{\mathbbm{1}\{x = T\}}\big((1 - \delta_j)\gamma_{Tx}\big)^{\mathbbm{1}\{x \neq T\}}\Big]^{\mathbbm{1}\{z_2 \geq 2\}}\bigg]^{\mathbbm{1}\{j - z_1 = 35\}}\nonumber\\
                            &\,\times\bigg[\Big[ \delta_j^{\mathbbm{1}\{x = A\}}\big((1 - \delta_j)\gamma_{Ax}\big)^{\mathbbm{1}\{x \neq A\}}\Big]^{\mathbbm{1}\{z_2 \% 2 = 0\}}\Big[ \delta_j^{\mathbbm{1}\{x = G\}}\big((1 - \delta_j)\gamma_{Gx}\big)^{\mathbbm{1}\{x \neq G\}}\Big]^{\mathbbm{1}\{z_2 \% 2 = 1\}}\bigg]^{\mathbbm{1}\{j - z_1 = 42\}}
        \end{align}
        

            
        \item Redefine $p_{jx}(\bm{z})$ derived above as $p'_{jx}(\bm{z})$.  When $z_1 = 4$, then $p_{jx}(\bm{z}) = q_{Sx}$.  Then our new likelihood equation is
            \begin{align*}
                p_{jx}(\bm{z}) = (q_{Sx})^{\mathbbm{1}\{z_1 = 4\}}p'_{jx}(\bm{z})^{\mathbbm{1}\{z_1 \neq 4\}}
            \end{align*}

        Define $\bm{X} = \{X_1, X_2, \dots, X_M\}$, and let $m_i$ be the length of sequence $X_i$.  Let $\bm{\theta} = \{\bm{q}_B, \bm{q}_S, \bm{\gamma}, \bm{\delta}\}$.  Then the complete data likelihood function is
                \begin{align}
                \mathcal{L}_c(\theta\mid \bm{X}, \bm{z}) &= \prod_{i = 1}^M\prod_{j = 1}^{m_i}\prod_{k = 0}^4\prod_{l = 0}^3[P(X_{ij} \mid \theta, \bm{z_i}) P(z_{i1} = k) P(z_{i2} = l)]^{\mathbbm{1}\{z_{i1} = k, z_{i2} = l\}} 
                \end{align}
        Since $z_2$ is undefined when $z_1 = 4$, we rearrange the above equation by factoring out the $z_1 = 4$ case:
                \begin{align}
                    \mathcal{L}_c(\theta\mid \bm{X}, \bm{z}) = &\prod_{i = 1}^M\Bigg[\bigg[\bigg(\prod_{j = 1}^{m_i}P(X_{ij}\mid \bm{\theta}, z_{i1} = 4)\bigg)P(z_{i1} = 4)\bigg]^{\mathbbm{1}\{z_{i1} = 4\}}\nonumber\\
                    &\ \,\times\bigg[\prod_{k = 0}^3\prod_{l = 0}^3\bigg(\prod_{j = 1}^{m_i}P(X_{ij}\mid \bm{\theta}, \bm{z_i})\bigg)P(z_{i1} = k)P(z_{i2} = l)\bigg]^{\mathbbm{1}\{z_{i1} = k, z_{i2} = l\}}\Bigg]\nonumber\\
                    = &\prod_{i = 1}^M\Bigg[\bigg[\bigg(\prod_{j = 1}^{m_i}P_{jX_{ij}}(4, \text{na})\bigg)\xi_{1, 4}\bigg]^{\mathbbm{1}\{z_{i1} = 4\}}\bigg[\prod_{k = 0}^3\prod_{l = 0}^3\bigg[\bigg(\prod_{j = 1}^{m_i}P_{jX_{ij}}(k, l)\bigg)\xi_{1, k}\xi_{2, l}\bigg]\bigg]^{\mathbbm{1}\{z_{i1} = k, z_{i2} = l\}}\Bigg]
                \end{align}
                where $\xi_{1, k}$ the probability of $z_1 = k$, and $\xi_{2, l}$ is the probability of $z_{2} = l$.
        Then, the complete data log-likelihood is
            \begin{align}\label{eq:cd_loglik}
                l_c(\theta\mid \bm{X}, \bm{z}) &= \sum_{i = 1}^M\Bigg[\mathbbm{1}\{z_{i1} = 4\} \bigg( \ln \xi_{1, 4} +  \sum_{j = 1}^{m_i} \bigg[\ln P_{jX_{ij}}(4, \text{na})\bigg] \bigg)\nonumber\\
                &\quad+ \sum_{k = 0}^3\sum_{l = 0}^3\mathbbm{1}\{z_{i1} = k, z_{i2} = l\}\bigg( \ln \xi_{1, k} + \ln \xi_{2, l} + \sum_{j = 1}^{m_i} \bigg[\ln P_{jX_{ij}}(k, l)\bigg] \bigg)\Bigg]\nonumber\\
                &= \sum_{i = 1}^M\Bigg[\mathbbm{1}\{z_{i1} = 4\}\big(\ln{\xi_{1, 4}} + P_1\big)  + \sum_{k = 0}^3\sum_{l = 0}^3\mathbbm{1}\{z_{i1} = k, z_{i2} = l\}\bigg( \ln \xi_{1, k} + \ln \xi_{2, l} + P_2\bigg)\Bigg]
            \end{align}
        where $P_1 = \sum_{j = 1}^{m_i}\ln P_{jX_{ij}}(4, \text{na})$ and $P_2 = \sum_{j = 1}^{m_i}\ln P_{jX_{ij}}(k, l)$.  
        
    \item Taking the expectation of equation (\ref{eq:cd_loglik}), we have
            \begin{align}\label{eq:Q_1}
                Q(\bm{\theta} \mid \bm{\theta}^{(m)}) &= \sum_{i = 1}^M\Bigg[ \mathbbm{E}\big(\mathbbm{1}\{z_{i1} = 4\} \mid X_i ; \bm{\theta}^{(m)}\big)\big[\ln{\xi_{1, 4} + P_1}\big]\nonumber\\
                &\qquad+\sum_{k = 0}^3\sum_{l = 0}^3\mathbbm{E}\big(\mathbbm{1}\{z_{i1} = k, z_{i2} = l\} \mid X_i ; \bm{\theta}^{(m)}\big)\bigg[\ln{\xi_{1, k}} + \ln{\xi_{2, l}} + P_2\bigg]\Bigg]
            \end{align}
        For some sequence $X_i$ and nucleotide $b \in N$, define $n_{ib} = \sum_{j = 1}^{m_i}\mathbbm{1}\{X_{ij} = b\}$.  Then, applying Bayes' rule to $\mathbbm{E}\big(\mathbbm{1}\{z_{i1} = 4\} \mid X_i ; \bm{\theta}^{(m)}\big)$, we have
            \begin{align}\label{eq:bayes_1}
                \mathbbm{E}_{i1}=\mathbbm{E}\big(\mathbbm{1}\{z_{i1} = 4\}\mid X_i; \bm{\theta}^{(m)} \big) &= P(z_{i1} = 4\mid X_i, \bm{\theta}^{(m)})\nonumber
                    = \frac{P(X_i\mid z_{i1} = 4, \bm{\theta}^{(m)})P(z_{i1} = 4)}
                    {P(X_i,\theta^{(m)}) \nonumber}\\
            \end{align}
            \begin{align}
                P(X_i\mid z_{i1} = 4, \bm{\theta}^{(m)})P(z_{i1} = 4)=\bigg[\prod_{j = 1}^{m_i}q_{SX_{ij}}\bigg]\xi_{1, 4}^{(m)}= \bigg[\prod_{b \in N}\big(q_{Sb}\big)^{n_{ib}}\bigg]\xi_{1, 4}^{(m)}\nonumber\\
            \end{align}
        where: \\
        $ {P(X_i,\theta^{(m)}) \nonumber} = {P(X_i\mid z_{i1} = 4, \bm{\theta}^{(m)})P(z_{i1} = 4)+\sum_{k=0}^3\sum_{l=0}^3P\big(X_i \mid z_{i1} = k, z_{i2} = l; \bm{\theta}^{(m)} \big)P(z_{i1} = k, z_{i2} = l)\nonumber}$\\\\
        Using Bayes' rule on the other expectation in equation (\ref{eq:Q_1}) yields
            \begin{align}\label{eq:bayes_2}
                \mathbbm{E}_{ikl2}=\mathbbm{E}\big(\mathbbm{1}\{z_{i1} = k, z_{i2} = l\}\mid X_i; \bm{\theta}^{(m)}\big) &= P\big(z_{i1} = k, z_{i2} = l\mid X_i; \bm{\theta}^{(m)}\big) \nonumber\\
                    &=\frac{P\big(X_i \mid z_{i1} = k, z_{i2} = l; \bm{\theta}^{(m)} \big)P(z_{i1} = k, z_{i2} = l)\nonumber}
                    {{P(X_i,\theta^{(m)}) \nonumber}}
            \end{align}
            \begin{align}
                P\big(X_i \mid z_{i1} = k, z_{i2} = l; \bm{\theta}^{(m)} \big)P(z_{i1} = k, z_{i2} = l)\nonumber
                    &=\bigg[\prod_{j = 1}^{m_i}P_{jX_{ij}}(k, l; \bm{\theta}^{(m)})\bigg]P(z_{i1} = k)P(z_{i2} = l)\nonumber\\
                    &=\bigg[\prod_{j = 1}^{m_i}P_{jX_{ij}}(k, l; \bm{\theta}^{(m)})\bigg]\xi_{1, k}^{(m)}\xi_{2, l}^{(m)}
            \end{align}
        Substituting (\ref{eq:bayes_1}) and (\ref{eq:bayes_2}) into equation (\ref{eq:Q_1}), we have this Q function
            \begin{align}\label{eq:Q_2}
                Q(\bm{\theta} \mid \bm{\theta}) &= \sum_{i = 1}^M\Bigg[\Bigg(\mathbbm{E}_{i1}\bigg(\ln{\xi_{1, 4}}+P_1\bigg)\Bigg) + \sum_{k = 0}^3\sum_{l = 0}^3\Bigg(\mathbbm{E}_{ikl2}\bigg(\ln{\xi_{1, k}}+\ln{\xi_{2, l}} + P_2\bigg) \Bigg) \Bigg]\nonumber\\
            \end{align}
        Before taking partial derivatives of equation (\ref{eq:Q_2}), we observe that the parameters $\bm{\delta}, \bm{q}_S, \bm{q}_B, \bm{\gamma}, \xi_{1, k}$, and $\xi_{2, l}$ only appear in the $P_1$ and $P_2$ terms and that everything else is a constant with respect to those variables.  For a given sequence $X_i$ and expanding the $P_1$ term in equation (\ref{eq:Q_2}), we have
            \begin{align}\label{eq:P_1}
                \sum_{j = 1}^{m_i}P_1 = \sum_{j = 1}^{m_i}\big[\ln{q_{SX_{ij}}}\big] = \sum_{b \in N}\big[ n_{ib} \ln{q_{Sb}}\big]
            \end{align}
        Expanding the $P_2$ sum gives
            \begin{align}\label{eq:P_2_1}
                \sum_{j = 1}^{m_i} P_2 = \sum_{j = 1}^{m_i}\big[\ln{P_{jX_{ij}}(k, l)}\big]
            \end{align}
        For some nucleotide $b \in N$, define $N'_{b} = N \setminus {b}$.  Then, expanding $\sum_{j = 1}^{m_i}\ln{P_{jX_{ij}}}(k, l)$, we have
            \begin{align}\label{eq:sum_ln_p}
                \sum_{j = 1}^{m_i}\ln{P_{jX_{ij}}}(k, l) &= \sum_{j = 1}^k\sum_{b \in N}\mathbbm{1}\{x_{ij} = b\} \ln{q_{Bb}} \nonumber\\
                &+\sum_{j - k \in \mathcal{I_P}}\sum_{a \in N}\bigg[\mathbbm{1}\{X_{ij} = a\}\ln{\delta_j} + \sum_{b \in N'_a}\mathbbm{1}\{X_{ij} = b\}\big[ \ln{(1 - \delta_j)} + \ln_{\gamma_{ab}} \big]\bigg]\nonumber\\
                &+ \bigg[\sum_{j = 19 + k}^{27 + k}\sum_{b \in N}\mathbbm{1}\{X_{ij} = N\}\ln{q_{Bb}}\bigg] + \bigg[\sum_{j = 53 + k}^{m_i}\sum_{b \in N}\mathbbm{1}\{X_{ij} = b\}\ln{q_{Sb}}\bigg]\nonumber\\
                &+\mathbbm{1}\{l < 2\}\sum_{j=1}^{mi}\mathbbm{1}\{j=35+k\}\bigg[\mathbbm{1}\{X_{ij} = C\}\ln{\delta_{j}} + \sum_{M\epsilon\{A,G,T\}}\mathbbm{1}\{X_{ij=M}\}\big(\ln{(1 - \delta_{1}) + \ln{\gamma_{CM}}}\big)\bigg]\nonumber\\
                &+\mathbbm{1}\{l \geq 2\}\sum_{j=1}^{mi}\mathbbm{1}\{j=35+k\}\bigg[\mathbbm{1}\{X_{ij} = T\}\ln{\delta_{j}} + \sum_{M\epsilon\{A,C,G\}}\mathbbm{1}\{X_{ij=M }\}\big(\ln{(1 - \delta_{1}) + \ln{\gamma_{TM}}}\big)\bigg]\nonumber\\
                &+\mathbbm{1}\{l\,\mathrm{mod}\,2 = 0\}\sum_{j=1}^{mi}\mathbbm{1}\{j=42+k\}\bigg[\mathbbm{1}\{X_{ij} = A\}\ln{\delta_{j}} + \sum_{M\epsilon\{C,G,T\}}\mathbbm{1}\{X_{ij=M }\}\big(\ln{(1 - \delta_{1}) + \ln{\gamma_{AM}}}\big)\bigg]\nonumber\\
                &+\mathbbm{1}\{l\,\mathrm{mod}\,2 = 1\}\sum_{j=1}^{mi}\mathbbm{1}\{j=42+k\}\bigg[\mathbbm{1}\{X_{ij} = G\}\ln{\delta_{j}} + \sum_{M\epsilon\{A,C,T\}}\mathbbm{1}\{X_{ij=M }\}\big(\ln{(1 - \delta_{1}) + \ln{\gamma_{GM}}}\big)\bigg]\nonumber\\
                &+ \lambda_B(1 - q_{BA}- q_{BC}- q_{BG} - q_{BT}) + \lambda_S(1 - q_{SA}- q_{SC}- q_{SG} - q_{ST})\nonumber\\
                &+ \lambda_{\gamma_A}(1 - \sum_{b \in N'_A}\gamma_{Ab}) + \lambda_{\gamma_C}(1 - \sum_{b \in N'_C}\gamma_{Cb})+ \lambda_{\gamma_G}(1 - \sum_{b \in N'_G}\gamma_{Gb})+ \lambda_{\gamma_T}(1 - \sum_{b \in N'_T}\gamma_{Tb})\nonumber\\
                &+ \lambda_{\xi_{1, k}}(1 - \xi_{1, 0} - \xi_{1, 1} - \xi_{1, 2} - \xi_{1, 3} - \xi_{1, 4}) + \lambda_{\xi_{2, l}}(1 - \xi_{2, 0} - \xi_{2, 1} - \xi_{2, 2} - \xi_{2, 3})\nonumber
            \end{align}
        where each of the $\lambda$ terms are Lagrange multipliers with corresponding constraints. We proceed to take partial derivatives of $Q$ with respect to each of the variables: $q_{Bb}, q_{Sb}, \gamma_{MN}, \delta_j, \xi_1k, \xi_2l$.  Lets start with $q_{Bb}, q_{Sb}$. For some nucleotide $b \in N$, we have:
        \begin{align}
            \frac{\partial Q}{\partial q_{Bb}}&=\sum_{i=1}^M\sum_{k=0}^3\sum_{l=0}^3\mathbbm{E}_{ikl2}\bigg[\sum_{j=1}^{k}\frac{\mathbbm{1}\{x_{ij}=b\}}{q_{Bb}}+\sum_{j=k+19}^{k+27}\frac{\mathbbm{1}\{x_{ij}=b\}}{q_{Bb}}  \bigg]-\lambda_B=0\nonumber\\
            \Rightarrow q_{Bb}&=\frac{1}{\lambda_B}\sum_{i=1}^M\sum_{k=0}^3\sum_{l=0}^3\mathbbm{E}_{ikl2}\bigg[\sum_{j=1}^{k}\mathbbm{1}\{x_{ij}=b\}+\sum_{j=k+19}^{k+27}\mathbbm{1}\{x_{ij}=b\}  \bigg]
        \end{align}
        Similarly, for $Q_{Sb}$, we have:
        \begin{align}
            \frac{\partial Q}{\partial q_{Sb}}&=\sum_{i=1}^{M}\bigg[\mathbbm{E}_{i1}\frac{n_{ib}}{q_{Sb}}+\sum_{k=0}^3\sum_{l=0}^3\mathbbm{E}_{ikl2}\bigg[\sum_{j=53+k}^{m_i}\frac{\mathbbm{1}\{x_{ij}=b\}}{q_{Sb}}\bigg] \bigg]-\lambda_S=0\nonumber\\
            \Rightarrow q_{Sb}&=\frac{1}{\lambda_S}\sum_{i=1}^{M}\bigg[\mathbbm{E}_{i1}{n_{ib}}+\sum_{k=0}^3\sum_{l=0}^3\mathbbm{E}_{ikl2}\bigg[\sum_{j=53+k}^{m_i}\mathbbm{1}\{x_{ij}=b\}\bigg] \bigg]
        \end{align}
        For $k = 0, 1, 2,$ or $3$, the following holds:
        \begin{align}
            \frac{\partial Q}{\partial \xi_{1,k}}&=\sum_{i=1}^M\sum_{l=0}^3\frac{\mathbbm{E}_{ikl2}}{\xi_{1,k}}-\lambda_{\xi_1}=0\qquad\implies\qquad
             \xi_{1,k}=\frac{1}{\lambda_{\xi_{1}}}\sum_{i=1}^M\sum_{l=0}^3\mathbbm{E}_{ikl2}
        \end{align}
        For $k = 4$, we have the following:
        \begin{align}
            \frac{\partial Q}{\partial \xi_{1,4}}&=\sum_{i=0}^M\bigg[\frac{\mathbbm{E}_{i1}}{\xi_{1,4}}\bigg] - \lambda_{\xi_{1}}= 0\qquad\implies\qquad
            \xi_{1,4}=\sum_{i=1}^M\frac{\mathbbm{E}_{i1}}{\lambda_{\xi_1}}
        \end{align}
        For $l = 0, 1, 2,$ or $3$, the following holds:
        \begin{align}
            \frac{\partial Q}{\partial \xi_{2,l}}&=\sum_{i=1}^M\sum_{k=0}^3\frac{\mathbbm{E}_{ikl2}}{\xi_{2,l}}-\lambda_{\xi_2}=0\qquad\implies\qquad
            \xi_{20}=\frac{1}{{\lambda_{\xi_{2}}}}{\sum_{i=1}^M\sum_{k=0}^3\mathbbm{E}_{ikl2}}
        \end{align}
        For $\delta_j$:
        \begin{align}
            \frac{\partial Q}{\partial\delta_j} &= \sum_{i=1}^M\sum_{k=0}^3\sum_{l=0}^3\mathbbm{E}_{ikl2}\Bigg[\mathbbm{1}\{j - z_1 \in \mathcal{I}_P\}\sum_{M\in N}\bigg[\frac{\mathbbm{1}\{X_{ij}=M\}}{\delta_j}-\sum_{N\neq M}\frac{\mathbbm{1}\{X_{ij}=N\}}{1-\delta_j} \bigg]\\\nonumber
            &+\mathbbm{1}\{l<2\}\bigg[\mathbbm{1}\{j=35+k\}\bigg[\frac{\mathbbm{1}\{X_{ij}=C\}}{\delta_j} - \sum_{M \in (A,C,T)}\frac{\mathbbm{1}\{X_{ij}=M\}}{1-\delta_j} \bigg]\bigg] \\\nonumber
            &+\mathbbm{1}\{l \geq 2\}\bigg[\mathbbm{1}\{j=35+k\}\bigg[\frac{\mathbbm{1}\{X_{ij}=T\}}{\delta_j} - \sum_{M \in (A,C,G)}\frac{\mathbbm{1}\{X_{ij}=M\}}{1-\delta_j} \bigg] \bigg]\\\nonumber
            &+\mathbbm{1}\{l\,\mathrm{mod}\,2 = 0\}\bigg[\mathbbm{1}\{j=42+k\}\bigg[\frac{\mathbbm{1}\{X_{ij}=A\}}{\delta_j} - \sum_{M \in (C,G,T)}\frac{\mathbbm{1}\{X_{ij}=M\}}{1-\delta_j} \bigg] \bigg]\\\nonumber
            &+\mathbbm{1}\{l\,\mathrm{mod}\,2 = 1\}\bigg[\mathbbm{1}\{j=42+k\}\bigg[\frac{\mathbbm{1}\{X_{ij}=G\}}{\delta_j} - \sum_{M \in (A,C,T)}\frac{\mathbbm{1}\{X_{ij}=M\}}{1-\delta_j} \bigg] \bigg]\Bigg]=0\\\nonumber
        \end{align}
        Define $A$ as the following:
        \begin{align}
            A = \,&\sum_{i=1}^{M}\sum_{k=0}^3\sum_{l=0}^3\mathbbm{E}_{ikl2}\Bigg[\mathbbm{1}\{j - k \in \mathcal{I}_P\}\sum_{M \in N}\mathbbm{1}\{x_{ij}=M\}\nonumber\\\nonumber
            &+\mathbbm{1}\{j=35+k\}\bigg[\mathbbm{1}\{l<2\}\mathbbm{1}\{x_{ij}=C\}+\mathbbm{1}\{l\geq2\}\mathbbm{1}\{x_{ij}=T\}\bigg]\nonumber\\\nonumber
            &+\mathbbm{1}\{j=42+k\}\bigg[\mathbbm{1}\{l\,\mathrm{mod}\,2 = 0\}\mathbbm{1}\{x_{ij}=A\} + \mathbbm{1}\{l\,\mathrm{mod}\,2 = 1\}\mathbbm{1}\{x_{ij}=G\} \bigg]\Bigg]\\\nonumber
        \end{align}
        Define $B$ as the following:
        \begin{align}
            B = &\,\sum_{i=1}^{M}\sum_{k=0}^3\sum_{l=0}^3\mathbbm{E}_{ikl2}\Bigg[\mathbbm{1}\{j - k \in \mathcal{I}_P\}\sum_{M \in (A,C,G,T)}\sum_{N \neq M}\mathbbm{1}\{x_{ij}=N\}\\\nonumber
            &+\mathbbm{1}\{j=35+k\}\bigg[\mathbbm{1}\{l<2\}\sum_{M \in (A,G,T)}\mathbbm{1}\{x_{ij}=M\}+\mathbbm{1}\{l\geq2\}\sum_{M \in (A,C,G)}\mathbbm{1}\{x_{ij}=M\} \bigg]\nonumber\\
            &+\mathbbm{1}\{j=42+k\}\bigg[\mathbbm{1}\{l\,\mathrm{mod}\,2 = 0\}(\sum_{M \in (C,G,T)}\mathbbm{1}\{x_{ij}=M\} \nonumber\\
            &+ \mathbbm{1}\{l\,\mathrm{mod}\,2 = 1\}\bigg(\sum_{M \in (A,C,T)}\mathbbm{1}\{x_{ij}=M\}\bigg) \bigg]\Bigg]\nonumber
        \end{align}
        Then, from (17)
        \begin{align}
            \frac{A}{\delta}-\frac{B}{1-\delta}=0 \qquad\Rightarrow \qquad A-\delta A=B\delta = A = \delta(A+B)\qquad \Rightarrow \qquad\delta_j=\frac{A}{A+B}
        \end{align}
        And finally for $\gamma$ we will take two examples, one for each value of $l$, which will consider one of the positions 35+k or 42+k. Then for $\gamma_{AC}$
        \begin{align}
            \frac{\partial Q}{\partial \gamma_{AC}} = \,&\sum_{i=1}^M\Bigg[\sum_{k=0}^3\sum_{l=0}^3\mathbbm{E}_{ikl2}\bigg[\sum_{j - k_1 \in \mathcal{I}_P}\bigg[\mathbbm{1}\{x_{ij}=C\}\bigg(\frac{1}{\gamma_{AC}} \bigg) \bigg] \bigg]\nonumber\nonumber\\
            &+ \mathbbm{1}\{l\,\mathrm{mod}\,2 = 0\}\sum_{j=1}^{m_i}\mathbbm{1}\{j=42+k\}\bigg[\mathbbm{1}\{x_{ij}=C\}\frac{1}{\gamma_{AC}} \bigg]\Bigg] - \lambda_{\gamma_A} = 0 \nonumber
        \end{align}
        Rearranging the above, we have:
        \begin{align}
            %
            \gamma_{AC} = \frac{\sum_{i=1}^M\Bigg[\sum_k\sum_l\mathbbm{E}_{ikl2}\bigg[\sum_{j - k_1 \in \mathcal{I}_P}\bigg[\mathbbm{1}\{x_{ij}=C\}\bigg] + \mathbbm{1}\{l\,\mathrm{mod}\,2 = 0\}\sum_{j=1}^{m_i}\mathbbm{1}\{j=42+k\}\bigg[\mathbbm{1}\{x_{ij}=C\} \bigg]\bigg]\Bigg]}{\lambda_{\gamma_A}}
        \end{align}
       
       Similarly for $\gamma_{TG}$
        
        \begin{align}
            \frac{\partial Q}{\partial \gamma_{TG}} = \,&\sum_{i=1}^M\Bigg[\sum_{k=0}^3\sum_{l=0}^3\mathbbm{E}_{ikl2}\bigg[\sum_{j - k_1 \in \mathcal{I}_P}\bigg[\mathbbm{1}\{x_{ij}=G\}\bigg(\frac{1}{\gamma_{TG}} \bigg) \bigg] \bigg]\nonumber\nonumber\\
            &+ \mathbbm{1}\{l\geq2\}\sum_{j=1}^{m_i}\mathbbm{1}\{j=35+k\}\bigg[\mathbbm{1}\{x_{ij}=G\}\frac{1}{\gamma_{AC}} \bigg]\Bigg] - \lambda_{\gamma_T} = 0 \nonumber
        \end{align}
        Rearranging the above, we have:
        \begin{align}
            %
            \gamma_{TG} = \frac{\sum_{i=1}^M\Bigg[\sum_k\sum_l\mathbbm{E}_{ikl2}\bigg[\sum_{j - k_1 \in \mathcal{I}_P}\bigg[\mathbbm{1}\{x_{ij}=G\}\bigg] + \mathbbm{1}\{l\geq2\}\sum_{j=1}^{m_i}\mathbbm{1}\{j=35+k\}\bigg[\mathbbm{1}\{x_{ij}=G\} \bigg]\Bigg]}{\lambda_{\gamma_T}} 
        \end{align}
        
        With this we can continue with the programming part of the assignment

    \end{enumerate}




\end{document}